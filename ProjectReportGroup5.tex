\title{Analysis of Hillary Clinton Emails}
\author{
	Debojyoti Dey \\
	Roll No. 15511264
	\and
	Nimisha Agarwal\\
	Roll No. 15511267
}
\date{\today}

\documentclass[12pt]{article}

\begin{document}
	\maketitle
\section{Problem Statement and Motivation}
The project is based on Hillary Clinton emails data-set. It consists of several tasks which are described below. A system is built to:
\begin{enumerate}
	\item Categorize emails topic-wise. In present scenario, most of the communications are done through emails. It results in large number of emails in email client used by a person. So, categorizing mails topic-wise organizes the mails in email client and makes the email client more user-friendly (For ex: Gmail). Topic models are used for this purpose that uncovers the hidden thematic structure in document collections. These algorithms helps in developing new ways to search, browse and summarize large archives of texts.
	\item Find relation of US with different nations in Hillary's perspective. Analysing sentiments of Hillary towards other countries can give an insight of US relations with them. Sentiment analysis is used to solve this purpose. It is the process of finding the overall contextual polarity of a document. It has various applications but mostly used in social networking websites like Twitter to analyse sentiment of a tweet and e-commerce websites for recommending products to a customer based on her reviews on other products.
	\item Determine the people linked to Hillary and her sentiments about them. Most frequently contacted persons can be found by the count of mails exchanged between Hillary and them. For the second part, sentiment analysis can be used. But the emails can be on any topic in general and may not be about that person at all. So, this task is not possible to achieve just by analysing the emails dataset.
\end{enumerate}
\section{Methodology}
\section{Results}
\section{Conclusion}
\section{References}
	
\end{document}