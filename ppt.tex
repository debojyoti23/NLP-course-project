\documentclass{beamer}

\mode<presentation> {

% The Beamer class comes with a number of default slide themes
% which change the colors and layouts of slides. Below this is a list
% of all the themes, uncomment each in turn to see what they look like.

%\usetheme{default}
%\usetheme{AnnArbor}
%\usetheme{Antibes}
%\usetheme{Bergen}
%\usetheme{Berkeley}
%\usetheme{Berlin}
%\usetheme{Boadilla}
\usetheme{CambridgeUS}
%\usetheme{Copenhagen}
%\usetheme{Darmstadt}
%\usetheme{Dresden}
%\usetheme{Frankfurt}
%\usetheme{Goettingen}
%\usetheme{Hannover}
%\usetheme{Ilmenau}
%\usetheme{JuanLesPins}
%\usetheme{Luebeck}
%\usetheme{Madrid}
%\usetheme{Malmoe}
%\usetheme{Marburg}
%\usetheme{Montpellier}
%\usetheme{PaloAlto}
%\usetheme{Pittsburgh}
%\usetheme{Rochester}
%\usetheme{Singapore}
%\usetheme{Szeged}
%\usetheme{Warsaw}

% As well as themes, the Beamer class has a number of color themes
% for any slide theme. Uncomment each of these in turn to see how it
% changes the colors of your current slide theme.

%\usecolortheme{albatross}
%\usecolortheme{beaver}
%\usecolortheme{beetle}
%\usecolortheme{crane}
%\usecolortheme{dolphin}
%\usecolortheme{dove}
%\usecolortheme{fly}
%\usecolortheme{lily}
%\usecolortheme{orchid}
%\usecolortheme{rose}
%\usecolortheme{seagull}
%\usecolortheme{seahorse}
%\usecolortheme{whale}
%\usecolortheme{wolverine}

%\setbeamertemplate{footline} % To remove the footer line in all slides uncomment this line
\setbeamertemplate{footline}[page number] % To replace the footer line in all slides with a simple slide count uncomment this line

%\setbeamertemplate{navigation symbols}{} % To remove the navigation symbols from the bottom of all slides uncomment this line
}

\usepackage{graphicx} % Allows including images
\usepackage{booktabs} % Allows the use of \toprule, \midrule and \bottomrule in tables

%----------------------------------------------------------------------------------------
%	TITLE PAGE
%----------------------------------------------------------------------------------------

\title[Short title]{Analysis of mails from Hillary Clinton’s Mail dataset} % The short title appears at the bottom of every slide, the full title is only on the title page

\author{Debojyoti Dey (15511264), Nimisha Agarwal (15511267)} % Your name
%\author{Debojyoti Dey}
%\institute[IITK] % Your institution as it will appear on the bottom of every slide, may be shorthand to save space
%{
%	Indian Institute of Technology Kanpur \\ % Your institution for the title page
%	\medskip
%	\textit{nimisha@cse.iitk.ac.in} % Your email address
%}
\date{\today} % Date, can be changed to a custom date

\setbeamertemplate{caption}[numbered]{}
\begin{document}
	
\begin{frame}
	\titlepage % Print the title page as the first slide
\end{frame}

\begin{frame}
	\frametitle{Overview} % Table of contents slide, comment this block out to remove it
	\tableofcontents % Throughout your presentation, if you choose to use \section{} and \subsection{} commands, these will automatically be printed on this slide as an overview of your presentation
\end{frame}

%----------------------------------------------------------------------------------------
%	PRESENTATION SLIDES
%----------------------------------------------------------------------------------------

%------------------------------------------------
\section{What we proposed?}
%------------------------------------------------

%------------------------------------------------

\begin{frame}
	\frametitle{Problem Statement}
	\begin{itemize}
		\item Extracting important topics from the e-mails.
		\item Getting an idea of US Foreign policy and relation with different nations.
		\item Persons linked to Clinton and Hillary's sentiment about them.
	\end{itemize}
\end{frame}

%------------------------------------------------

\begin{frame}
	\frametitle{Sentiment Analysis}
	\begin{itemize}
		\item Process of identifying and categorizing opinion.
		\item For determining the person's attitude towards a particular thing.
		\item Various libraries are available in python for this purpose.
		\item We have used VADER library in NLTK for this purpose.
	\end{itemize}
\end{frame}

%------------------------------------------------

\begin{frame}
	\frametitle{Topic Modelling}
	\begin{itemize}
		\item Technique for discovering the abstract "topics" in a set of documents.
		\item If a document is on a certain topic, then it is expected to contain particular words.
		\item We have used LDA library of gensim for this purpose.
	\end{itemize}
\end{frame}

%------------------------------------------------

%------------------------------------------------
\section{What have we achieved?}
%------------------------------------------------

%------------------------------------------------

\begin{frame}
	\frametitle{}
	
\end{frame}

%------------------------------------------------

%------------------------------------------------
\section{How we achieved them?}
%------------------------------------------------

%------------------------------------------------
\section{Conclusion}
%------------------------------------------------

%------------------------------------------------
\section{References}
%------------------------------------------------



\begin{frame}
	\frametitle{}
	
\end{frame}

%------------------------------------------------

\begin{frame}
	\frametitle{}
	
\end{frame}

%------------------------------------------------

\begin{frame}
	\frametitle{}
	
\end{frame}

%------------------------------------------------

\begin{frame}
	\frametitle{}
	
\end{frame}

%------------------------------------------------

\begin{frame}
	\frametitle{}
	
\end{frame}

%------------------------------------------------

\begin{frame}
	\frametitle{}
	
\end{frame}

%------------------------------------------------

\begin{frame}
	\frametitle{}
	
\end{frame}

%------------------------------------------------

\begin{frame}
	\frametitle{}
	
\end{frame}

%------------------------------------------------

\begin{frame}
	\frametitle{}
	
\end{frame}

%------------------------------------------------

\begin{frame}
	\frametitle{}
	
\end{frame}

%------------------------------------------------

\begin{frame}
	\frametitle{}
	
\end{frame}

%------------------------------------------------

\begin{frame}
	\frametitle{}
	
\end{frame}

%------------------------------------------------

\begin{frame}
	\frametitle{}
	
\end{frame}

%------------------------------------------------

\begin{frame}
	\frametitle{}
	
\end{frame}

%------------------------------------------------

\begin{frame}
	\frametitle{}
	
\end{frame}

%------------------------------------------------

\begin{frame}
	\frametitle{}
	
\end{frame}

%------------------------------------------------

\begin{frame}
	\frametitle{}
	
\end{frame}

%------------------------------------------------

\begin{frame}
	\frametitle{}
	
\end{frame}

%------------------------------------------------

\begin{frame}
	\frametitle{}
	
\end{frame}

%------------------------------------------------

\begin{frame}
	\frametitle{}
	
\end{frame}

%------------------------------------------------

\begin{frame}
	\frametitle{}
	
\end{frame}

%------------------------------------------------

\begin{frame}
	\frametitle{}
	
\end{frame}

%------------------------------------------------

\begin{frame}
	\frametitle{}
	
\end{frame}

%------------------------------------------------

\begin{frame}
	\frametitle{}
	
\end{frame}

%------------------------------------------------

\begin{frame}
	\frametitle{}
	
\end{frame}

%------------------------------------------------

\begin{frame}
	\frametitle{}
	
\end{frame}

%------------------------------------------------

\begin{frame}
	\frametitle{}
	
\end{frame}

%------------------------------------------------

\begin{frame}
	\frametitle{}
	
\end{frame}

%------------------------------------------------

\begin{frame}
	\frametitle{}
	
\end{frame}

%------------------------------------------------


%----------------------------------------------------------------------------------------

\end{document}